% !TeX program = lualatex

\documentclass{beamer}

\usepackage{hyperref}
\usepackage{booktabs}
\usepackage{fontspec}
\usepackage{microtype}
\usepackage{palatino}
\usepackage[default]{FiraSans}

\setmonofont{PragmataPro Mono}

\mode<presentation>{
  \usetheme{default}
  \usefonttheme{structurebold}

  \definecolor{structureFG}{HTML}{5C6166}
  \definecolor{structureBG}{HTML}{FCFCFC}
  \definecolor{alert}{HTML}{E65050}
  \definecolor{example}{HTML}{6CBF43}

  \setbeamercolor*{strucure}{fg=structureFG,bg=structureBG}%
  \setbeamercolor*{local strucure}{fg=structureFG,bg=structureBG}%
  \setbeamercolor*{titlelike}{fg=structureFG}%
  \setbeamercolor*{alerted text}{fg=alert}%
  \setbeamercolor*{example text}{fg=example}%

  \setbeamercolor*{caption name}{fg=structureFG}%
  \setbeamercolor*{block title}{fg=structureFG}%
  \setbeamercolor*{itemize item}{fg=structureFG}%
  \setbeamercolor*{itemize subitem}{fg=structureFG}%
  \setbeamercolor*{itemize subsubitem}{fg=structureFG}%
  \setbeamercolor*{enumerate item}{fg=structureFG}%
  \setbeamercolor*{enumerate subitem}{fg=structureFG}%
  \setbeamercolor*{enumerate subsubitem}{fg=structureFG}%

  \setbeamercolor*{navigation symbols}{fg=structureFG!50}
  \setbeamercolor*{navigation symbols dimmed}{fg=structureFG!15}

  \setbeamertemplate{footline}[frame number]
  \setbeamertemplate{headline}{}
}

\usepackage{fancyvrb}
\usepackage{listings}
\lstset{
  inputencoding=utf8,
  basicstyle=\linespread{1.0}\ttfamily\mdseries,
  fancyvrb=true,
  sensitive=true,
  breaklines=true,
  extendedchars=false,
  showstringspaces=false,
  columns=fixed,
  stepnumber=1,
  escapeinside={\%*}{*)},
  lineskip=-5pt,
  basewidth=0.65em
}

\definecolor{pc}{HTML}{55B4D4}
\definecolor{branching}{HTML}{FA8D3E}
\definecolor{arithmetic}{HTML}{ED9366}
\definecolor{stack}{HTML}{A37ACC}
\definecolor{userio}{HTML}{4CBF99}
\definecolor{gridio}{HTML}{86B300}
\definecolor{stringmode}{HTML}{FF7383}
\definecolor{endprogram}{HTML}{E65050}
\definecolor{ignore}{HTML}{787B80}
\definecolor{number}{HTML}{5C6166}
\definecolor{codebg}{HTML}{F3F4F5}

\colorlet{stringbg}{stringmode!20}
\colorlet{pcbg}{pc!20}
\colorlet{branchingbg}{branching!20}

\lstdefinelanguage{befunge}{
  basicstyle=\ttfamily\mdseries\color{ignore},
  literate=%
    {^}{{{\color{pc}\^{}}}}1
    {<}{{{\color{pc}<}}}1
    {>}{{{\color{pc}>}}}1
    {v}{{{\color{pc}v}}}1
    {?}{{{\color{pc}?}}}1
    {\#}{{{\color{pc}\#}}}1
    {|}{{{\color{branching}|}}}1
    {\_}{{{\color{branching}\_}}}1
    {+}{{{\color{arithmetic}+}}}1
    {-}{{{\color{arithmetic}-}}}1
    {*}{{{\color{arithmetic}*}}}1
    {/}{{{\color{arithmetic}/}}}1
    {\%}{{{\color{arithmetic}\%}}}1
    {!}{{{\color{arithmetic}!}}}1
    {\`}{{{\color{arithmetic}\`}}}1
    {:}{{{\color{stack}:}}}1
    {\\}{{{\color{stack}\textbackslash}}}1
    {\$}{{{\color{stack}\$}}}1
    {.}{{{\color{userio}.}}}1
    {,}{{{\color{userio},}}}1
    {&}{{{\color{userio}&}}}1
    {\~}{{{\color{userio}\~}}}1
    {g}{{{\color{gridio}g}}}1
    {p}{{{\color{gridio}p}}}1
    {"}{{{\color{stringmode}"}}}1
    {@}{{{\color{endprogram}@}}}1
    {0}{{{\color{number}0}}}1
    {1}{{{\color{number}1}}}1
    {2}{{{\color{number}2}}}1
    {3}{{{\color{number}3}}}1
    {4}{{{\color{number}4}}}1
    {5}{{{\color{number}5}}}1
    {6}{{{\color{number}6}}}1
    {7}{{{\color{number}7}}}1
    {8}{{{\color{number}8}}}1
    {9}{{{\color{number}9}}}1
}
\newcommand{\befunge}[1]{\text{\lstinline[basicstyle=\ttfamily\mdseries\small,language=befunge]!#1!}}
\lstMakeShortInline[columns=fixed,language=befunge]§
\newcommand{\stringStyle}[1]{{\lstinline[basicstyle=\ttfamily\mdseries\color{stringmode}]!#1!}}

\lstdefinelanguage{brainfuck}{
  basicstyle=\ttfamily\mdseries\color{ignore},
  literate=%
    {<}{{{\color{pc}<}}}1
    {>}{{{\color{pc}>}}}1
    {[}{{{\color{branching}[}}}1
    {]}{{{\color{branching}]}}}1
    {+}{{{\color{arithmetic}-}}}1
    {-}{{{\color{arithmetic}-}}}1
    {.}{{{\color{userio}.}}}1
    {,}{{{\color{userio},}}}1
}

%% TikZ ist kein Zeichenprogramm
\usepackage{tikz}
\usetikzlibrary{arrows.meta}
\usetikzlibrary{patterns}
\usetikzlibrary{shapes.symbols}
\usetikzlibrary{shapes.multipart}
\usetikzlibrary{shapes.misc}
\usetikzlibrary{shadows}
\usetikzlibrary{shadings}
\usetikzlibrary{calc}
\usetikzlibrary{decorations.pathreplacing}
%% layered TiKZ drawings
\pgfdeclarelayer{background}
\pgfdeclarelayer{foreground}
\pgfsetlayers{background,main,foreground}
%% TikZ-based (scatter) plots
\usepackage{pgfplots}
%% read table cells from CSV input
\usepackage{pgfplotstable}
\usepackage{pgffor}
\usetikzlibrary{fpu}
%% make sure that TikZ's calc library and lstlisting's 'mathescape=true' cooperate
\makeatletter
\global\let\tikz@ensure@dollar@catcode=\relax
\makeatother

%% lengths that measure character box width and height in code/in a listing
\newlength{\x}
\newlength{\y}
\newlength{\xx}


\title[Befunge]{Befunge-93 in SQL}
\subtitle{Abusing SQLs Turing Completeness}
\author[Tim F.]{Tim Fischer}
\institute[TUE]{Eberhard Karls Universität Tübingen \\ \smallskip \textit{t.fischer@student.uni-tuebingen.de}}
\date[\today]{SQL is a Programming Language \\ \today}

\begin{document}

\begin{frame}
  \titlepage
\end{frame}

\begin{frame}
  \frametitle{Befunge TL;DR}

  \begin{columns}[c]
    \begin{column}{0.45\textwidth}
      Befunge is a...
      \begin{itemize}
        \item stack-based
        \item Turing complete
        \item two-dimensional
        \item self-modifying
        \item imperative
        \item esoteric
        \item programming language.
      \end{itemize}
    \end{column}
  \end{columns}
\end{frame}

\begin{frame}[fragile]
  \frametitle{Befunge — Brainfuck but somehow weirder...}

  \begin{figure}[t]
    \centering
    \tiny
    %% measure the width of xx in a listing
    \settowidth{\xx}{\lstinline[columns=fixed]{xx}}\setlength{\x}{0.5\xx}
    \setlength{\y}{2ex}
    \begin{tikzpicture}[x=\x,y=-\y,inner sep=0mm]
      %% check width/height of character boxes
      % \draw[help lines,xstep=1,ystep=-1] (0,0) grid (70,7);
      \node[anchor=north west] at (0,0) {%
        \begin{lstlisting}[language=befunge,gobble=10]
          222p882**1+11p>133p                    >33g1+33p 22g33g-v>22g33g%#v_v
           o                                                      >|
            2                              v,,,,, ,,,.g22"is prime"<
             1                             >   v ^                          <
                        ^_@#-g11g22p22+1g22,*25<,,,,,,,,,,,,.g22"is not prime"<
        \end{lstlisting}%
      };
      \begin{pgfonlayer}{background}
        \fill[codebg,rounded corners=1pt] (-1  ,-0.5) rectangle (70,6);
      \end{pgfonlayer}
    \end{tikzpicture}
  \end{figure}

\end{frame}

% \begin{frame}
%   \frametitle{Befunge — Brainfuck but somehow weirder...}

%   Sed iaculis \alert{dapibus gravida}. Morbi sed tortor erat, nec interdum arcu. Sed id lorem lectus. Quisque viverra augue id sem ornare non aliquam nibh tristique. Aenean in ligula nisl. Nulla sed tellus ipsum. Donec vestibulum ligula non lorem vulputate fermentum accumsan neque mollis.

%   \bigskip % Vertical whitespace

%   % Quote example
%   \begin{quote}
%     Sed diam enim, sagittis nec condimentum sit amet, ullamcorper sit amet libero. Aliquam vel dui orci, a porta odio.\\
%     --- Someone, somewhere\ldots
%   \end{quote}

%   \bigskip % Vertical whitespace

%   Nullam id suscipit ipsum. Aenean lobortis commodo sem, ut commodo leo gravida vitae. Pellentesque vehicula ante iaculis arcu pretium rutrum eget sit amet purus. Integer ornare nulla quis neque ultrices lobortis.
% \end{frame}

% \begin{frame}
%   \frametitle{Lists}
%   \framesubtitle{Bullet Points and Numbered Lists} % Optional subtitle

%   \begin{itemize}
%     \item Lorem ipsum dolor sit amet, consectetur adipiscing elit
%     \item Aliquam blandit faucibus nisi, sit amet dapibus enim tempus
%           \begin{itemize}
%             \item Lorem ipsum dolor sit amet, consectetur adipiscing elit
%             \item Nam cursus est eget velit posuere pellentesque
%           \end{itemize}
%     \item Nulla commodo, erat quis gravida posuere, elit lacus lobortis est, quis porttitor odio mauris at libero
%   \end{itemize}

%   \bigskip % Vertical whitespace

%   \begin{enumerate}
%     \item Nam cursus est eget velit posuere pellentesque
%     \item Vestibulum faucibus velit a augue condimentum quis convallis nulla gravida
%   \end{enumerate}
% \end{frame}

% \begin{frame}
%   \frametitle{Blocks of Highlighted Text}

%   \begin{block}{Block Title}
%     Lorem ipsum dolor sit amet, consectetur adipiscing elit. Integer lectus nisl, ultricies in feugiat rutrum, porttitor sit amet augue.
%   \end{block}

%   \begin{exampleblock}{Example Block Title}
%     Aliquam ut tortor mauris. Sed volutpat ante purus, quis accumsan.
%   \end{exampleblock}

%   \begin{alertblock}{Alert Block Title}
%     Pellentesque sed tellus purus. Class aptent taciti sociosqu ad litora torquent per conubia nostra, per inceptos himenaeos.
%   \end{alertblock}

%   \begin{block}{} % Block without title
%     Suspendisse tincidunt sagittis gravida. Curabitur condimentum, enim sed venenatis rutrum, ipsum neque consectetur orci.
%   \end{block}
% \end{frame}

% \begin{frame}
%   \frametitle{Multiple Columns}
%   \framesubtitle{Subtitle} % Optional subtitle

%   \begin{columns}[c] % The "c" option specifies centered vertical alignment while the "t" option is used for top vertical alignment
%     \begin{column}{0.45\textwidth} % Left column width
%       \textbf{Heading}
%       \begin{enumerate}
%         \item Statement
%         \item Explanation
%         \item Example
%       \end{enumerate}
%     \end{column}
%     \begin{column}{0.5\textwidth} % Right column width
%       Lorem ipsum dolor sit amet, consectetur adipiscing elit. Integer lectus nisl, ultricies in feugiat rutrum, porttitor sit amet augue. Aliquam ut tortor mauris. Sed volutpat ante purus, quis accumsan dolor.
%     \end{column}
%   \end{columns}
% \end{frame}

\end{document}